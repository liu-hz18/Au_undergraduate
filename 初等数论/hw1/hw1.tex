\documentclass[UTF8, onecolumn, a4paper]{article}
\usepackage{ctex}
\usepackage{appendix}
\usepackage{geometry}
\usepackage{amsmath, amsthm}
\usepackage{multirow, multicol}
\usepackage{subfigure}
\usepackage{float}
\usepackage{graphicx}
\usepackage{lettrine}
\usepackage{authblk}
\usepackage{xcolor, fontspec}%用于设置颜色
\usepackage[ruled,vlined]{algorithm2e}
\usepackage{listings}%用于显示代码
\usepackage{amsfonts}
\usepackage{amssymb}
\geometry{left=3.0cm,right=3.0cm,top=2.0cm,bottom=3.0cm}

\title{\textbf{Elementary Number Theory: Homework1}}%———总标题
\author{刘泓尊\quad 2018011446\quad 计84 }
\begin{document}
\maketitle

\section*{Exercises 1.1}
\subsection*{2. 证明: 如果$a$和$b$为正整数,则在所有形式为$a - bk(k\in \mathbb{Z})$的正整数中有一个最小元.}
\begin{proof}
令$S = \{a - bk \mid a\in \mathbb{N^+}, b\in \mathbb{N^+}, a-bk\in\mathbb{N^+}\}$.\\
$\because a - b(-1) = a+b\in S$\\
$\therefore S \neq \Phi$\\
故$S$是非空的正整数集合,由良序公理, $S$有最小元.
\end{proof}

\subsection*{32. 证明强狄利克雷逼近定理}
\begin{proof}
考虑$\{j\alpha \mid 0 \leq j\alpha < 1, j = 0, 1, 2, \cdots , n+1\}$将区间$[0,1)$等分为长度为$1 / (n+1)$的区间$\left[\frac{k-1}{n+1}, \frac{k}{n+1}\right), k = 1, 2, \cdots , n+1$. \\
因此我们有$n+2$个数, $n+1$个区间, 根据鸽巢原理, 必有一个区间包含至少两个数. \\即$\exists r, s \in \mathbb{Z}, , s.t.|\{r\alpha\} - \{s\alpha\}| \leq 1 / (n+1)$.\\
令$a = s - r, b = [s\alpha] - [r\alpha]$.\\
$\because 0\leq r < s \leq n+1$\\
$\therefore 1 \leq a \leq n$\\
$\therefore |a\alpha - b| = |(s-r)\alpha - ([s\alpha] - [r\alpha])| = |(s\alpha - [s\alpha]) - (r\alpha - [r\alpha])| = |\{s\alpha\} - \{r\alpha\}| < 1/(n+1)$
故$\exists a(\in [1, n]), b,\quad s.t.|a\alpha - b| \leq 1/(n+1)$
\end{proof}
\subsection*{43. 证明存在无穷多个乌拉姆数}
\begin{proof}
假设只存在有限个乌拉姆数, 所有乌拉姆数构成集合$U$.\\
根据定义,$|U| \geq 2$\\
取其中最大的两个数$u_{n-1}, u_n$, 则$u_n+u_{n-1}$是乌拉姆数, 因为$u_i + u_j < u_n + u_{n-1}, \forall j < n, i < n, i\neq j$\\
假设不成立, 因此有无穷多个乌拉姆数.
\end{proof}
\subsection*{44. 证明$e$是无理数}
\begin{proof}
假设$e\in \mathbb{Q}$, 即$e = a / b, a\in \mathbb{Z},b\in \mathbb{Z}, b\neq0$\\
令$k \geq b$, $c = k!(e - 1 - 1/1! - 1/2! - 1/3! - \cdots 1/k!)$\\
因为括号中分母都可整除$k!$, 所以$c\in \mathbb{Z}$\\
$\because e = 1 + 1/1! + 1/2! + \cdots$\\
$\therefore 0 < c = k!(1/(k+1)! + 1/(k+2)! + \cdots) = 1/(k+1) + 1/(k+1)(k+2) + \cdots < 1/(k+1) + 1/(k+1)^2 + \cdots = 1/k $\\
$\therefore 0 < c < 1/k, c \notin \mathbb{Z}$\\
矛盾!故$e$是无理数.
\end{proof}
\subsection*{45. 证明实数集不可数.}
\begin{proof}
使用$Cantor's$ $Diagonal$ $Argument$给出证明:\\
假设实数集$\mathbb{R}$是可数集, 则$S = \{a \mid a\in \mathbb{R}, 0 < a < 1\}\subset \mathbb{R}\ $可数.\\
故存在双射$f: \mathbb{Z^+}\rightarrow (0,1)$, 设$f(k) = a_k\in(0,1), k\in \mathbb{Z^+}$\\
因为$a_k\in(0, 1)$, 所以$a_k$可表示为$0. a_{k1} a_{k2} a_{k3}\cdots$, 其中$a_{ki}$是$a_k$的第$i$位小数.\\
对于实数$c = c_1c_2c_3\cdots$, 若$a_{kk} = 5$, 则$c_k = 4$;若$a_{kk} \neq 5$, 则$c_k = 5$. 那么$c\neq a_k = f(k), \forall k\in \mathbb{Z^+}$, 所以$f$不是$\mathbb{Z^+}\rightarrow (0,1)$的双射.\\
所以实数集不可数. 
\end{proof}
\section*{Exercises 1.3}
\subsection*{15. 使用数学归纳法证明$H_{2^n} \geq 1 + n/2$}
\begin{proof}
基础: $n= 0$时, $H_{2^0} = H_1 = 1 \geq 1 = 1 + 0/2$\\
假设: 对于$n$有$H_{2^n} \geq 1 + n/2$成立.\\
递推: $H_{2^{n+1}} = \sum_{j=1}^{2^n}1/j + \sum_{j=2^n+1}^{2^{n+1}}1/j \geq H_{2^n} + \sum_{j=2^n+1}^{2^{n+1}}1/2^{n+1}\geq 1 + n/2 + 2^n\cdot 1/2^{n+1} = 1 + n/2 + 1/2 = 1 + (n+1)/2.$\\
所以$H_{2^n} \geq 1 + n/2$.
\end{proof}
\subsection*{16. 使用数学归纳法证明$H_{2^n} \leq 1 + n$}
\begin{proof}
基础: $n= 0$时, $H_{2^0} = H_1 = 1 \leq 1 = 1 + 0$\\
假设: 对于$n$有$H_{2^n} \leq 1 + n$成立.\\
递推: $H_{2^{n+1}} = \sum_{j=1}^{2^n}1/j + \sum_{j=2^n+1}^{2^{n+1}}1/j \leq H_{2^n} + \sum_{j=2^n+1}^{2^{n+1}}1/2^n\leq 1 + n + 2^n\cdot 1/2^n = 1 + n + 1 = 1 + (n+1).$\\
所以$H_{2^n} \leq 1 + n$.
\end{proof}
\section*{Exercises 1.4}
\subsection*{10. 证明: $f_{2n+1} = f_{n+1}^2 + f_{n}^2$, $f_0 = 0$}
\begin{proof}
基础: $n=0$时, $f_3 = 2 = f_2^2 + f_1^2 = 1 + 1$; $n=2$时, $f_5 = 5 = f_3^2 + f_2^2 = 2^2 + 1^2$.\\
假设: $f_{2k-1} = f_{k}^2 + f_{k-1}^2, \forall k \leq n$\\
递推: $f_{2k+1} = f_{2k} + f_{2k-1} = 2f_{2k-1} + f_{2k-2} = 2f_{2k-1} + f_{2k-1} - f_{2k-3} = 3f_{2k-1} - f_{2k-3} = 3(f_k^2 + f_{k-1}^2) - (f_{k-1}^2 + f_{k-2}^2) = 3f_k^2 + 2f_{k-1}^2 - (f_k - f_{k-1})^2 = 2f_k^2 + f_{k-1}^2 + 2f_kf_{k-1} = 2f_k^2 + (f_{k+1} - f_k)^2 + 2f_k(f_{k+1} - f_k) = f_{k+1}^2 + f_k^2$\\
故$f_{2n+1} = f_{n+1}^2 + f_{n}^2$
\end{proof} 
\subsection*{11. 证明: $f_{2n} = f_{n+1}^2 - f_{n-1}^2$, $f_0 = 0$}
\begin{proof}
$n = 1$时: $f_2 = 1 = f_2^2 - f_0^2 = 1 - 0 = 1$\\
由上题结论: $f_{2n+1} = f_{n+1}^2 + f_{n}^2$\\
$n\geq 2$时, $f_{2n} = f_{2n-1} + f_{2n-2} = 2f_{2n-1} - f_{2n-3} = 2(f_{n}^2 + f_{n-1}^2) - (f_{n-1}^2 + f_{n-2}^2) = 2f_n^2 + f_{n-1}^2 - f_{n-2}^2 = 2f_n^2 + f_{n-1}^2 - (f_n^2 - f_{n-1})^2 =f_n^2 + 2f_n f_{n-1}= (f_{n+1} - f_{n-1})^2 + 2(f_{n+1} - f_{n-1})f_{n-1} = f_{n+1}^2 - f_{n-1}^2$\\
故$f_{2n} = f_{n+1}^2 - f_{n-1}^2$
\end{proof}
\subsection*{12. 证明: $f_n + f_{n-1} + f_{n-2} + 2f_{n-3} + 4f_{n-4} +8f_{n-5} + \cdots + 2^{n-3} = 2^{n-1}$}
\begin{proof}
令$S_n = f_n + f_{n-1} + f_{n-2} + 2f_{n-3} + 4f_{n-4} +8f_{n-5} + \cdots + 2^{n-3}f_1$\\
使用数学归纳法:\\
基础: $S_3 = f_3 + f_2 + f_1 = 2 + 1 + 1 = 4 = 2^{3 - 1}$, $S_4 = f_4 + f_3 + f_2  + 2f_1 = 3 + 2 + 1 + 2\times1 = 8 = 2^{4 - 1}$\\
假设: $S_k = 2^{k-1}, \forall k \leq n$\\
递推: $S_{n+1} = f_{n+1} + f_{n} + f_{n-1} + 2 f_{n-2} + 4f_{n-3} +\cdots = (f_{n} + f_{n-1}) + (f_{n-1} + f_{n-2}) + (f_{n-2} + f_{n-3}) + \cdots = (f_n + f_{n-1} + f_{n-2} + 2f_{n-3} + \cdots) + (f_{n-1} + f_{n-2} + f_{n-3} + 2f_{n-4}+\cdots) + 2^{n-2}f_1 = S_n + S_{n-1} + 2^{n-2} = 2^{n-1} + 2^{n-2} + 2^{n-2} = 2^n$\\
根据归纳假设,  $S_n = f_n + f_{n-1} + f_{n-2} + 2f_{n-3} + 4f_{n-4} +8f_{n-5} + \cdots + 2^{n-3} = 2^{n-1}$.
\end{proof}

\subsection*{34.$F = \begin{pmatrix}1& 1 \\1 &0 \end{pmatrix}$, 证明$n\in \mathbb{Z}$时, $F^n = \begin{pmatrix}f_{n+1}& f_{n} \\f_{n} &f_{n-1}\end{pmatrix}$}
\begin{proof}
$n = 1$时:
$$F^1 = \begin{pmatrix}1& 1 \\1 &0 \end{pmatrix} = \begin{pmatrix}f_2& f_1 \\f_1 &f_0 \end{pmatrix}$$
假设: 结论对$n$成立.\\
递推:
$$F^{n+1} = FF^n = \begin{pmatrix}1& 1 \\1 &0 \end{pmatrix} \begin{pmatrix}f_{n+1}& f_{n} \\f_{n} &f_{n-1} \end{pmatrix} = \begin{pmatrix}f_{n+1} + f_n& f_{n} + f_{n-1}\\ f_{n+1}&f_{n} \end{pmatrix} = \begin{pmatrix}f_{n+2}& f_{n+1} \\f_{n+1} &f_{n} \end{pmatrix}$$
故$F^n = \begin{pmatrix}f_{n+1}& f_{n} \\f_{n} &f_{n-1}\end{pmatrix}$
\end{proof}

\subsection*{40. 若$a_n = \frac{1}{\sqrt{5}}(\alpha^n - \beta^n)$, 其中$\alpha = \frac{1+\sqrt{5}}{2}, \beta = \frac{1-\sqrt{5}}{2}$, 则$a_n = a_{n-1} + a_{n-2}$, 且$a_1 = a_2 = 1$, 从而得到$f_n = a_n$, 即第$n$个裴波那契数.}
\begin{proof}
易证$\alpha^2 = \alpha+1$, $\beta^2 = \beta + 1$.\\
$a_1 = \frac{1}{\sqrt{5}}(\alpha - \beta) = \frac{1}{\sqrt{5}}\left(\frac{1+\sqrt{5}}{2} - \frac{1-\sqrt{5}}{2}\right) = 1 = f_1$\\
$a_2 = \frac{1}{\sqrt{5}}(\alpha^2 - \beta^2) = \frac{1}{\sqrt{5}}\left((\frac{1+\sqrt{5}}{2})^2 - (\frac{1-\sqrt{5}}{2})^2\right) = 1 = f_2$\\
$a_{n-2} + a_{n-1} = \frac{1}{\sqrt{5}} \left(\alpha^{n-1} - \beta^{n-1}\right)+\frac{1}{\sqrt{5}}\left(\alpha^{n-2} - \beta^{n-2}\right)= \frac{1}{\sqrt{5}}\left(\alpha^{n-2}(\alpha+1) - \beta^{n-2}(\beta+1)\right) = \frac{1}{\sqrt{5}}\left(\alpha^{n-2}\alpha^2 - \beta^{n-2}\beta^2\right) = \frac{1}{\sqrt{5}}\left(\alpha^n - \beta^n\right) = a_n$\\
即$a_n = a_{n-1} + a_{n-2}$\\
所以$f_n = a_n$
\end{proof}

\section*{Exercises 1.5}
\subsection*{53. 证明当$n\in \mathbb{N^+}\cup\{0\}$时, $\left[(2 + \sqrt{3})^n\right]$是奇数}
\begin{proof}
根据二项式定理:
$(2 + \sqrt{3})^n + (2 - \sqrt{3})^n = \sum_{j=0}^{n}\binom{n}{j}2^j\sqrt{3}^{n-j} + \sum_{j=0}^{n}\binom{n}{j}2^j(-1)^{n-j}\sqrt{3}^{n-j} = 2(2^n + \binom{n}{2}3\cdot 2^{n-2} + \binom{n}{4}3^2\cdot2^{n-4} + \cdots) = 2l, l\in \mathbb{Z}$\\
即$(2 + \sqrt{3})^n + (2 - \sqrt{3})^n$是偶数.\\
又$(2 - \sqrt{3}) < 1$, $\left[(2 + \sqrt{3})^n\right] = (2 + \sqrt{3})^n + (2 - \sqrt{3})^n - 1$\\
所以$\left[(2 + \sqrt{3})^n\right]$是奇数.
\end{proof}
\end{document}