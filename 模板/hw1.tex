\documentclass[UTF8, onecolumn, a4paper]{article}
\usepackage{ctex}
\usepackage{geometry}
\usepackage{amsmath}
\usepackage{multirow, multicol}
\usepackage{subfigure}
\usepackage{float}
\usepackage{graphicx}
\usepackage{lettrine}
\usepackage{authblk}
\usepackage{xcolor}%用于设置颜色
\usepackage{fontspec}
\geometry{left=3.0cm,right=3.0cm,top=2.0cm,bottom=3.0cm}

\title{\textbf{Intermediate Macroeconomics PS1}}%———总标题
\author{刘泓尊\quad 2018011446\quad 计84 \\ \texttt{liu-hz18@mails.tsinghua.edu.cn} }
%\affil{Department of Computer Science, Tsinghua University}
\date{2020.02}

\begin{document}
\maketitle

\section{What counts as GDP?}
\begin{description}
\item[a.]不计入GDP。因为该汽车是使用过的, 在其生产并第一次售出之后就已经算入了GDP。
\item[b.]车子的价格不计入GDP。理由同a;但是如果dealer提供的服务价值也包含在了\$3000中,则服务的价值计入GDP。
\item[c.]计入GDP。是政府当年支出,属于GDP。
\item[d.]计入GDP。助教提供了服务,属于GDP。
\item[e.]计入GDP。是在国内进行的活动。
\item[f.]不计入GDP。房屋是去年建造的,已经计入了去年的GDP,对当年没有贡献。
\end{description}

\section{Analysis}
\begin{description}
\item[a.]\quad
\begin{description}
\item[1960:]($15482.71\times180.67) / (121.19 \times 667.07) = 3445.93\%$
\item[1980:]$(26113.12\times227.22) / (220.68 \times 981.24) = 2740.10\%$	\item[2014:]$(46405.25\times318.86) / (3862.92 \times 1364.27) = 280.82\%$
\end{description}
可见: 2014年美国总GDP与中国总GDP的比例缩小了10倍以上,所以中国正在追赶上美国。
\item[b.]1960-1980:\quad
\begin{description}
\item[America:]$(26113.12\times227.22) / (25482.71\times180.67) = (1 + x)^{20}$\\
解得$x = 3.83\%$ 
\item[China:]$(220.68\times981.24) / (121.19\times667.07) = (1+x)^{20}$\\
解得$x = 5.05\%$
\end{description}
\item[c.]\quad
\begin{description}
\item[America 2014:]$(26113.12\times227.22)\times(1+3.83\%)^34 = \$21,295,321.051$ million\\(actual: $\$14,796,778.015$ million)
\item[China 2014:]($220.68\times981.24)\times(1+5.05\%)^34 = \$1,156,123.285$ million\\(actual: $\$5,270,065.868$ million)
\end{description}
\item[d.]$220.68\times981.24\times(1+5.05\%)^n = 227.22\times26113.12\times(1+3.83\%)^n$\\解得$n = 283.40$年\\
故,预计中国会在$1980+283.40\approx2263$年赶上美国。
\item[e.]1980-2014:\quad
\begin{description}
\item[America:]$(46405.25\times318.86) / (26113.12\times227.22) = (1 + x)^{34}$\\
解得$x = 2.724\%$ 
\item[China:]$(3862.92\times1364.27) / (220.68\times981.24) = (1+x)^{34}$\\
解得$x = 9.843\%$\\
$(3862.92\times1364.27)\times(1+9.843\%)^n = (46405.25\times318.86)\times(1+2.724\%)^n$\\解得$n = 15.41$年\\
故,预计中国会在$2014+15.41\approx 2029$年赶上美国。
\end{description}
\item[f.]\quad
\begin{center}
\begin{tabular}{cccccc}
	\hline
	\textbf{Rate} && \multicolumn{2}{c}{GDP per capita} & \multicolumn{2}{c}{Population}  \\
	\cline{2-6}
	&& US & China & US & China \\
	\hline
	1960-1980&& 2.648\% & 3.042\% &  1.153\% & 1.948\% \\
	1980-2014&& 2.917\%& 15.387\% & 1.709\% & 1.661\%  \\
	\hline
\end{tabular}
\end{center}
从表中可以看到,中国在1980-2014年人均GDP的增长率(15.387\%)显著高于美国同期指标,且明显高于中国的人口增长率,所以中美GDP总量的差距缩小主要是因为{\bfseries 中国人均GDP的增长}.
\end{description}

\end{document}
