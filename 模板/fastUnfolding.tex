
%\documentclass[twocolumn]{article} %双栏设置, 或者用 逗号 隔开
\documentclass[UTF8, onecolumn, a4paper]{article}
\usepackage{ctex}
\usepackage{geometry}
\usepackage{amsmath}
%\usepackage{hyperref}
\usepackage{multirow, multicol}
\usepackage{subfigure}
\usepackage{float}
\usepackage{graphicx}
%\usepackage[ruled]{algorithm2e}
\usepackage[ruled,vlined]{algorithm2e}
%\usepackage[linesnumbered,boxed]{algorithm2e}
\usepackage{ctex}
\usepackage{lettrine}
\usepackage{authblk}
\usepackage{listings}%用于显示代码
\usepackage{xcolor}%用于设置颜色
\usepackage{fontspec}
\geometry{left=3.0cm,right=3.0cm,top=2.0cm,bottom=3.0cm}

\title{Community Detection}%———总标题
\author{Liu Hongzun}
\affil{Department of Computer Science, Tsinghua University}
\date{2020.02.10}

\begin{document}
	\thanks{liu-hz18@mails.tsinghua.edu.cn}
	\maketitle %输入此句才会显示标题
	\setlength{\columnsep}{25pt}% 调整文本距中心线的边距
	\columnseprule=0.7pt %实现插入分隔线
	\begin{multicols}{2}% 分栏
		\lettrine[lines=2]{S}{ ocial}, technological and information systems can often be described in terms of complex
		networks that have a topology of interconnected nodes combining organization and
		randomness.The typical size of large networks such as social network services, mobile phone networks or the web now counts in millions when not billions of nodes
		and these scales demand new methods to retrieve comprehensive information from their
		structure.A promising approach consists in decomposing the networks into sub-units
		or communities, which are sets of highly inter-connected nodes. The identification of
		these communities is of crucial importance as they may help to uncover a-priori unknown
		functional modules such as topics in information networks or cyber-communities in social
		networks. Moreover, the resulting meta-network, whose nodes are the communities, may
		then be used to visualize the original network structure.
		The problem of community detection requires the partition of a network into
		communities of densely connected nodes, with the nodes belonging to different
		communities being only sparsely connected. Precise formulations of this optimization
		problem are known to be computationally intractable. Several algorithms have therefore
		been proposed to find reasonably good partitions in a reasonably fast way. This search
		for fast algorithms has attracted much interest in recent years due to the increasing
		availability of large network data sets and the impact of networks on every day life. One
		can distinguish several types of community detection algorithms: divisive algorithms
		detect inter-community links and remove them from the network, agglomerative
		algorithms merge similar nodes/communities recursively and optimization methods
		are based on the maximisation of an objective function.
	\end{multicols}

	定义图$G$的模块度(modularity):
	\begin{equation}
	\begin{aligned}
	Q &= \frac{1}{2m}\underset{i,j\in 1,2,...n}{\sum}\left(A_{ij} - \frac{d_i d_j}{2m}\right)\delta(C_i, C_j)\\
	&=\underset{i}{\sum}\left(\frac{d_{in}(C_i)}{2m} - \left(\frac{d_{tot}(C_i)}{2m}\right)^2\right)
	\end{aligned}
	\end{equation}
	将节点$i$从社群$C_k$移出,移入社群$C_j$,对模块度$Q$的贡献为:
	\begin{equation}
	\begin{aligned}
	\triangle Q(i, C_j) &= \frac{1}{2m} \left[ \left(d_{in}(C_j) + d(k, C_j) - \frac{(d_{tot}(C_j)+d_k)^2}{2m}\right)+
	\left( d_{in}(C_i) + d(k, C_i) - \frac{(d_{tot}(C_i)-d_k)^2}{2m} \right)\right]\\ 
	&\qquad - \frac{1}{2m} \left[ \left(d_{in}(C_j) - \frac{d_{tot}(C_j)^2}{2m}\right) + 
	\left( d_{in}(C_i) - \frac{ d_{tot} (C_i)^2 }{ 2m } \right) \right] \\
	&= \frac{1}{2m}\left[d(C_j, k) - d(C_i, k) - \frac{2}{2m}\left(d_k \cdot(d_{tot}(C_j) - d_{tot}(C_i) + d_k)\right)\right]
	\end{aligned}
	\end{equation}
	
	下面我们给出图的社群发现算法之一: Fast Unfolding\\
	\begin{algorithm}[H]
		\caption{Fast Unfolding Method}%算法名字
		%\LinesNumbered %要求显示行号
		\KwIn{$G=(V, E)$;}%输入参数
		\KwOut{clustering of $G$;}%输出
		$k = 0, G^0 = G$;\\ %\;用于换行
		\While{community partition (modularity) can still be changed (improved)}{
			make a simple clustering $C^k$ of $G^k$ such that $C_i^k = \{i\}$;\\
			\For{node $i \in G^k$}{
				remove the node $i$ from its community $C_i^k$;\\
				$C_N$ = set of neighbour communities of node $i$;\\
				$C_j^k = \arg\underset{C_{j'}^k\in C_N}{\max}\triangle Q(i, C_{j'}^k)$;\\
				%argmax是一种函数,是对函数求参数(集合)的函数。当我们有另一个函数y=f(x)时,
				%若有结果x0= argmax(f(x)),则表示当函数f(x)取x=x0的时候,得到f(x)取值范围的最大值
				\eIf{$\triangle Q(i, C_{j'}^k)>0$}
				{
					add node $i$ to the community $C_j^k$;
				}
				{
					leave node $i$ in the community $C_i^k$;
				}
			}
			build a new graph $G^{k+1}$ whose nodes are the communities of $C^k$;\\
			$k = k+1$;
		}
		make a clustering $C_{final}$ of $G$;\\
		\Return $C_{final}$
	\end{algorithm}
	This approach reduces the computation time, especially on large networks, and still provides a high coefficient of modularity.\\
	
	\lstset{%代码块全局设置
		backgroundcolor=\color{red!3!green!3!blue!3},%代码块背景色为浅灰色
		rulesepcolor= \color{gray}, %代码块边框颜色
		breaklines=true,  %代码过长则换行
		numbers=left, %行号在左侧显示
		numberstyle= \small,%行号字体
		%keywordstyle= \color{red},%关键字颜色
		commentstyle=\color{gray}, %注释颜色
		frame=shadowbox,%用方框框住代码块
		xleftmargin=1em,
		xrightmargin=0em,
		tabsize=5,
		%rulesepcolor=\color{red!20!green!20!blue!20},  %阴影颜色
		keywordstyle={\color{blue!90!}\fontspec{Consolas Bold}},   %关键字颜色
		commentstyle={\color{blue!70!black}\fontspec{Consolas Italic}},   %注释颜色
		stringstyle=\color{orange!100!black}, %字符串颜色
		numberstyle=\color{purple}, %行号颜色
		%basicstyle=\ttfamily, %代码风格
		basicstyle=\fontspec{Consolas},
		showstringspaces=false,          % underline spaces within strings only  
		showtabs=false,
		captionpos=t, %文件标题位置
		flexiblecolumns
	}
	
\begin{lstlisting}[language={java}, title={helloWorld.java}]
public class Main {
	public static void main(String[] args){
		System.out.println("Hello World");
	}
}
\end{lstlisting}

%在代码块中插入公式使用:mathescape
\begin{lstlisting}[language={c++}, title={bubbleSort.c},mathescape] 
#include <iostream>  
#define LENGTH 8  
using namespace std;  
//测试用的代码,bubbleSort函数  
int main() {  
	int temp,number[LENGTH]={95, 45, 15, 78, 84, 51, 24, 12};  
	for(int i = 0; i < LENGTH; i++){  
		for(int j = 0; j < LENGTH - 1 - i; j++){  
			if(number[j] > number[j+1]) {  
				temp = number[j];  
				number[j] = number[j+1];  
				number[j+1] = temp;  
			} //if end  
		}
	}
	for(int i = 0; i < LENGTH; i++) 
		cout << number[i] << " ";  
	cout << endl;
	/* the following code computes $\sum_{i=0}^{n}i$ */
	for (i = 1; i <= limit; i++) {
		sum += i;
	}

	return 0;  
}
\end{lstlisting}

\begin{lstlisting}[language={python}, title={test.py}]
import random
n = 100
s = [0 for i in range(n)]

for j in range(1000):
count = [0 for m in range(n)]
i = 0
while i < n:
count[i] += 1
if random.randint(0, 1) == 0:
i += 1
for k in range(n):
s[k] += count[k]

print(s)
\end{lstlisting}
	
	\begin{algorithm}[H]
		\caption{How to write algorithms}
		\KwIn{this text}
		\KwOut{how to write algorithm with \LaTeX2e }
		initialization;\\
		\While{not at end of this document}{
			read current;\\
			\eIf{understand}
			{
				go to next section;
				current section becomes this one;
			}
			{
				go back to the beginning of current section;
			}
		}
	\end{algorithm}

	\begin{algorithm} 
		\SetAlgoNoLine  %去掉之前的竖线
		\caption{identify Row Context} 
		\KwIn{$r_i$, $Backgrd(T_i)$=${T_1,T_2,\ldots ,T_n}$ and similarity threshold $\theta_r$} 
		\KwOut{$con(r_i)$} 
		$con(r_i)= \Phi$\\ 
		\For{$j=1;j \le n;j \ne i$} 
		{ 
			float $maxSim=0$\\
			$r^{maxSim}=null$\\
			\While{not end of $T_j$} 
			{ 
				compute Jaro($r_i,r_m$)($r_m\in T_j$)\\
				\If{$(Jaro(r_i,r_m) \ge \theta_r)\wedge (Jaro(r_i,r_m)\ge r^{maxSim})$} 
				{ 
					replace $r^{maxSim}$ with $r_m$\\
				} 
			} 
			$con(r_i)=con(r_i)\cup {r^{maxSim}}$\\
		} 
		return $con(r_i)$\\
	\end{algorithm}
	
	%常用模板
	\begin{algorithm}[H]
		\caption{algorithm caption}%算法名字
		\LinesNumbered %要求显示行号
		\KwIn{input parameters A, B, C}%输入参数
		\KwOut{output result}%输出
		some description\; %\;用于换行
		\For{condition}{
			only if\\
			\If{condition}{
				1\\
			}
		}
		\While{not at end of this document}{
			if and else\\
			\eIf{condition}{
				1\\
			}{
				2\\
			}
		}
		\ForEach{condition}{
			\If{condition}{
				1\\
			}
		}
	\end{algorithm}
	

	
\end{document}

