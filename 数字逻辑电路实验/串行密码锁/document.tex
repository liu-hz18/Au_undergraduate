
\documentclass[UTF8, onecolumn, a4paper]{article}
\usepackage{ctex}
\setlength{\parindent}{2em}
\usepackage{appendix}
\usepackage{geometry}
\usepackage{amsmath, amsthm}
\usepackage{multirow, multicol}
\usepackage{subfigure}
\usepackage{float}
\usepackage{graphicx}
\usepackage{lettrine}
\usepackage{authblk}
\usepackage{indentfirst}
\usepackage{xcolor, fontspec}%用于设置颜色
\usepackage[ruled,vlined]{algorithm2e}
\usepackage{listings}%用于显示代码
\usepackage[colorlinks,
linkcolor=red,
anchorcolor=blue,
citecolor=green
]{hyperref}
\usepackage{tikz}
\usetikzlibrary{trees}
\geometry{left=2.5cm,right=2.5cm,top=2.0cm,bottom=2.0cm}


\title{\textbf{串行密码锁: 实验报告}}%———总标题
\author{刘泓尊\quad 2018011446\quad 计84}
%\affil{Department of Computer Science, Tsinghua University}

\begin{document}
\maketitle
\tableofcontents
\lstset{%代码块全局设置
	backgroundcolor=\color{red!3!green!3!blue!3},%代码块背景色为浅灰色
	rulesepcolor= \color{gray}, %代码块边框颜色
	breaklines=true,  %代码过长则换行
	numbers=left, %行号在左侧显示
	numberstyle= \small,%行号字体
	%keywordstyle= \color{red},%关键字颜色
	commentstyle=\color{gray}, %注释颜色
	frame=shadowbox,%用方框框住代码块
	xleftmargin=0em,
	xrightmargin=0em,
	tabsize=5,
	%rulesepcolor=\color{red!20!green!20!blue!20},  %阴影颜色
	keywordstyle={\color{blue!90!}\fontspec{Consolas Bold}},   %关键字颜色
	commentstyle={\color{blue!70!black}\fontspec{Consolas Italic}},   %注释颜色
	stringstyle=\color{orange!100!black}, %字符串颜色
	numberstyle=\color{purple}, %行号颜色
	%basicstyle=\ttfamily, %代码风格
	basicstyle=\fontspec{Consolas},
	showstringspaces=false,          % underline spaces within strings only  
	showtabs=false,
	captionpos=t, %文件标题位置
	flexiblecolumns
}
%\vspace*{20}
\section{File Structure}
\tikzstyle{every node}=[draw=black,thick,anchor=west]
\tikzstyle{selected}=[draw=red,fill=red!30]
\tikzstyle{optional}=[dashed,fill=gray!50]
\begin{center}
	\begin{tikzpicture}
	[
	grow via three points={one child at (0.5,-0.7) and
		two children at (0.5,-0.7) and (0.5,-1.4)},
	edge from parent path={(\tikzparentnode.south)  |-(\tikzchildnode.west)}]
	\node {2018011446}
	child { node {counter}
		child {node {decoder.vhd}}
		child {node {setpwd.vhd}}
		child {node {checkpwd.vhd}}
		child {node {lock.vhd}}
		child {node {Waveform.vwf}}	
	}
	child [missing] {}
	child [missing] {}
	child [missing] {}
	child [missing] {}					
	child [missing] {}
	child {node {lock.mp4}}
	child {node {lock.json}}
	child {node {串行密码锁仿真样例.png}}
	child { node {2018011446\_刘泓尊\_串行密码锁.pdf}};
	\end{tikzpicture}
\end{center}
\clearpage
\section{实验目的}
\begin{enumerate}
	\item[(1)] 学习使用状态机来控制电路工作,在不同的状态下完成相应的功能。
	\item[(2)] 进一步掌握时序逻辑电路的基本分析和设计方法。
	\item[(3)] 学会利用软件仿真实现对数字电路的逻辑功能进行验证和分析.
\end{enumerate}
\section{实验任务}
\begin{enumerate}
	\item[(1)] 设计一个4位16进制串行电子密码锁,功能包括:设置密码,验证密码。
	\item[(2)] 实现密码预置(管理员密码)和系统报警功能。
\end{enumerate}
	
\section{代码及注释}
在mode=“00”时是设置密码状态,mode=“01”时是验证密码状态。mode=“1x”时仍是验证密码,但是验证管理员密码。管理员密码我在程序中设置为“0000”。每次更换操作后均应按下“rst”,每输入一位之后均应按下“clk”.
\paragraph*{}
我使用状态机实现了串行密码锁的“设置密码”与“验证密码”。其流程与实验说明中的流程一致。我将程序模块分为“输入密码(setpwd)”与“验证密码(checkpwd)”两个模块,每个模块内的核心逻辑就是状态机。同时,为了便于直观显示,我将输入的每个数字通过“显示译码器”输出到7段数码管。
\subsection{设置密码}
设置密码的状态机与课本给出的一致。

\begin{lstlisting}[language={VHDL}, title={setpwd.vhd}]
entity setpwd is
	port(
		clk, rst: in std_logic;
		mode: in std_logic_vector(1 downto 0);
		code: in std_logic_vector(3 downto 0);
		pwd0, pwd1, pwd2, pwd3: out std_logic_vector(3 downto 0);
		currentnum: out std_logic_vector(3 downto 0)
	);
end setpwd;

architecture bhv_set of setpwd is
	signal state: integer := 0;
begin
	currentnum <= code;
	process(clk, rst) begin
		if rst = '1' then --异步复位
			state <= 1;
		elsif clk'event and clk = '1' then
			if mode = "00" then
				case state is--设置密码状态机
				when 1 =>
					pwd0 <= code; state <= 2;
				when 2 =>
					pwd1 <= code; state <= 3;
				when 3 =>
					pwd2 <= code; state <= 4;
				when 4 =>
					pwd3 <= code; state <= 0;
				when others => null;
				end case;
			end if;
		end if;
	end process;
end bhv_set;
\end{lstlisting}
\subsection{验证密码}
验证密码的流程与课本给出的一致,只是加了mode="01"还是mode="1x"的判断。
\begin{lstlisting}[language={VHDL}, title={checkpwd.vhd}]
entity checkpwd is
	port(
		clk, rst: in std_logic;
			mode: in std_logic_vector(1 downto 0);
			code: in std_logic_vector(3 downto 0);
			unlock, err, alarm: buffer std_logic;
			pwd0, pwd1, pwd2, pwd3: in std_logic_vector(3 downto 0);--4位16进制密码,由setpwd提供
			currentnum: out std_logic_vector(3 downto 0) -- 数码管输出当前数位
	);
	type fourbitpwd is array(3 downto 0) of integer; 
end checkpwd;

architecture bhv_check of checkpwd is
	signal state: integer := 0; --状态机当前状态
	signal cnt: integer := 0;--输入错误的次数
	constant adminpwd: fourbitpwd := (0, 0, 0, 0);--admin密码设为0000
begin
	currentnum <= code;
	process(clk, rst) begin
		if rst = '1' then --reset时alarm不清零
			state <= 1;--状态机:i状态开始接受第i位密码
			unlock <= '0'; err <= '0';
			if alarm = '1' then
				cnt <= 0;
			end if;
		elsif clk'event and clk='1' then
			if mode = "01" then --user验证密码模式
				case state is--状态机
				when 1 =>
					if code = pwd0 then
						state <= 2;
					else
						err <= '1'; cnt <= cnt + 1; state <= 0;--输入错误
					end if;
				when 2 =>
					if code = pwd1 then
						state <= 3;
					else
						err <= '1'; cnt <= cnt + 1; state <= 0;
					end if;
				when 3 =>
					if code = pwd2 then
						state <= 4;
					else 
						err <= '1'; cnt <= cnt + 1; state <= 0;
					end if;
				when 4 =>
					if code = pwd3 then --正确
						cnt <= 0; err <= '0'; unlock <= '1'; state <= 0;
					else
						err <= '1'; cnt <= cnt + 1; state <= 0;
					end if;
				when others => null;
				end case;
				if cnt > 1 then
					alarm <= '1';
				end if;
			elsif (mode = "10" or mode = "11") then --admin模式
				case state is
				when 1 =>
					if CONV_INTEGER(code) = adminpwd(3) then
						state <= 2;
					else
						err <= '1'; state <= 0;--输入错误
					end if;
				when 2 =>
					if CONV_INTEGER(code) = adminpwd(2) then
						state <= 3;
					else
						err <= '1'; state <= 0;
					end if;
				when 3 =>
					if CONV_INTEGER(code) = adminpwd(1) then
						state <= 4;
					else
						err <= '1'; state <= 0;
					end if;
				when 4 =>
					if CONV_INTEGER(code) = adminpwd(0) then
						cnt <= 0; err <= '0'; unlock <= '1'; 
						alarm <= '0';--警报关闭
						state <= 0; --正确
					else
						err <= '1'; state <= 0;
					end if;
				when others => null;
				end case;
				if cnt > 1 then
					alarm <= '1';
				end if;
			end if;
		end if;
	end process;
end bhv_check;
\end{lstlisting}
\subsection{串行密码锁}
串行密码锁的实现即例化了上述的“设置密码”与“验证密码”模块,同时对输入数字做了译码处理,用于七段数码管输出。
\begin{lstlisting}[language={VHDL}, title={lock.vhd}]
-- 串行密码锁:支持设置密码、验证密码、三次输入限制、管理员模式 --
entity lock is
	port(
		code: in std_logic_vector(3 downto 0); --only for one digit
		mode: in std_logic_vector(1 downto 0); --mode: 1x:Admin, 00:set password, 01:check password
		clk, rst: in std_logic;
		unlock_out, err_out, alarm_out: out std_logic; 
		curnum: out std_logic_vector(6 downto 0)--Show the input number, for debug;
	);
end lock;

architecture bhv of lock is
	component checkpwd
		port(
			clk, rst: in std_logic;
			mode: in std_logic_vector(1 downto 0);
			code: in std_logic_vector(3 downto 0);
			unlock, err, alarm: buffer std_logic;
			pwd0, pwd1, pwd2, pwd3: in std_logic_vector(3 downto 0);--4位16进制密码,由setpwd提供
			currentnum: out std_logic_vector(3 downto 0) -- 数码管输出当前数位
		);
	end component;
	component setpwd
		port(
			clk, rst: in std_logic;
			mode: in std_logic_vector(1 downto 0);
			code: in std_logic_vector(3 downto 0);
			pwd0, pwd1, pwd2, pwd3: out std_logic_vector(3 downto 0);
			currentnum: out std_logic_vector(3 downto 0)
		);
	end component;
	component decoder
		port(
			bit_4_vec: in std_logic_vector(3 downto 0);
			bit_7_vec: out std_logic_vector(6 downto 0)
		);
	end component;
	signal pwd0, pwd1, pwd2, pwd3: std_logic_vector(3 downto 0);
	signal setnum: std_logic_vector(3 downto 0);
	signal checknum: std_logic_vector(3 downto 0);
	signal tempnum: std_logic_vector(3 downto 0);
	signal alarm, err, unlock: std_logic;
begin
	cpwd: checkpwd port map(clk=>clk, rst=>rst, mode=>mode, code=>code, unlock=>unlock, err=>err, alarm=>alarm, pwd0=>pwd0, pwd1=>pwd1, pwd2=>pwd2, pwd3=>pwd3, currentnum=>checknum);--验证密码元件例化
	spwd: setpwd port map(clk=>clk, rst=>rst, mode=>mode, code=>code, pwd0=>pwd0, pwd1=>pwd1, pwd2=>pwd2, pwd3=>pwd3, currentnum=>setnum);--设置密码元件例化
	tempnum <= setnum when mode = "00" else checknum;--显示的数字
	alarm_out <= alarm;  err_out <= err;  unlock_out <= unlock;
	de: decoder port map(bit_4_vec=>tempnum, bit_7_vec=>curnum);--显示译码
end bhv;
\end{lstlisting}

\section{仿真结果}
使用Quartus的ModelSim进行仿真(附"\textbf{./lock/Waveform.vwf}"文件).下图详细说明了电路功能,包括“设置密码”,“验证密码”,“正确亮起unlock”,“错误亮起err”,“输入三次错误密码后亮起alarm”,“管理员模式下输入0000解除alarm”,“重置密码”等功能。
\begin{figure}[htb]
	\centering
	\includegraphics[width=1.1\textwidth]{串行密码锁说明.png}
\end{figure}

%\clearpage
\section{JieLabs运行结果(附录屏)}
\paragraph*{}
我在JieLabs上进行了硬件验证,并将实验结果录屏保存在./\textbf{lock.mp4}下,也随附平台导出的\textbf{lock.json}文件。
\paragraph*{}
该视频流程为: “设置密码为137F”$\rightarrow$“输入正确密码137F,unlock亮起”$\rightarrow$“输入错误密码,err亮起”$\rightarrow$“输入3次错误密码,alarm亮起”$\rightarrow$“输入管理员密码0000解锁,消除alarm”。下面是实验过程截图.
\begin{figure}[htb]
	\centering
	\includegraphics[width=0.6\textwidth]{jielabs.png}
\end{figure}
\paragraph*{}
说明: 左下角从左至右3个LED分别为“unlock, err, alarm”.左下角4个开关代表输入的1位16进制数,该数字也经过译码显示在7段数码管上。下层中部模块最左边两个开关为mode的控制,左侧高位右侧低位,为了直观,我将其显示在右上角最左侧数码管处。右下角模块左侧为rst,右侧为clk。
\section{实验总结}
\paragraph*{}
本次实验是我第一次遇到较为实用性的功能电路,我学习了状态机在时序逻辑电路中的应用,学到了buffer, case, integer等语句的使用,以及CONV\_INTEGER函数(用于将std\_logic\_vector转换为integer)。综合练习了所学知识,设计代码更加得心应手,收获良多。最后,感谢老师和助教在微信群的耐心答疑与帮助!
\end{document}