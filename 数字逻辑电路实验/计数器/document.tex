
\documentclass[UTF8, onecolumn, a4paper]{article}
\usepackage{ctex}
\setlength{\parindent}{2em}
\usepackage{appendix}
\usepackage{geometry}
\usepackage{amsmath, amsthm}
\usepackage{multirow, multicol}
\usepackage{subfigure}
\usepackage{float}
\usepackage{graphicx}
\usepackage{lettrine}
\usepackage{authblk}
\usepackage{indentfirst}
\usepackage{xcolor, fontspec}%用于设置颜色
\usepackage[ruled,vlined]{algorithm2e}
\usepackage{listings}%用于显示代码
\usepackage[colorlinks,
linkcolor=red,
anchorcolor=blue,
citecolor=green
]{hyperref}
\usepackage{tikz}
\usetikzlibrary{trees}
\geometry{left=2.5cm,right=2.5cm,top=2.0cm,bottom=2.0cm}


\title{\textbf{计数器的设计: 实验报告}}%———总标题
\author{刘泓尊\quad 2018011446\quad 计84}
%\affil{Department of Computer Science, Tsinghua University}

\begin{document}
\maketitle
\tableofcontents
\lstset{%代码块全局设置
	backgroundcolor=\color{red!3!green!3!blue!3},%代码块背景色为浅灰色
	rulesepcolor= \color{gray}, %代码块边框颜色
	breaklines=true,  %代码过长则换行
	numbers=left, %行号在左侧显示
	numberstyle= \small,%行号字体
	%keywordstyle= \color{red},%关键字颜色
	commentstyle=\color{gray}, %注释颜色
	frame=shadowbox,%用方框框住代码块
	xleftmargin=1em,
	xrightmargin=0em,
	tabsize=5,
	%rulesepcolor=\color{red!20!green!20!blue!20},  %阴影颜色
	keywordstyle={\color{blue!90!}\fontspec{Consolas Bold}},   %关键字颜色
	commentstyle={\color{blue!70!black}\fontspec{Consolas Italic}},   %注释颜色
	stringstyle=\color{orange!100!black}, %字符串颜色
	numberstyle=\color{purple}, %行号颜色
	%basicstyle=\ttfamily, %代码风格
	basicstyle=\fontspec{Consolas},
	showstringspaces=false,          % underline spaces within strings only  
	showtabs=false,
	captionpos=t, %文件标题位置
	flexiblecolumns
}
%\vspace*{20}
\section{File Structure}
\tikzstyle{every node}=[draw=black,thick,anchor=west]
\tikzstyle{selected}=[draw=red,fill=red!30]
\tikzstyle{optional}=[dashed,fill=gray!50]
\begin{center}
	\begin{tikzpicture}
	[
	grow via three points={one child at (0.5,-0.7) and
		two children at (0.5,-0.7) and (0.5,-1.4)},
	edge from parent path={(\tikzparentnode.south)  |-(\tikzchildnode.west)}]
	\node {2018011446}
	child { node {counter}
		child {node {d\_ff.vhd}}
		child {node {decoder.vhd}}
		child {node {clkpersecond.vhd}}
		child {node {fourbitcounter.vhd}}
		child {node {counter.vhd}}
		child {node {Waveform.vwf}}	
	}
	child [missing] {}
	child [missing] {}
	child [missing] {}
	child [missing] {}					
	child [missing] {}
	child [missing] {}
	child {node {counter-manual.mp4}}
	child {node {counter-auto.mp4}}
	child {node {counter-manual.json}}
	child {node {counter-auto.json}}
	child { node {2018011446\_刘泓尊\_计数器.pdf}};
	\end{tikzpicture}
\end{center}
\clearpage
\section{实验目的}
\begin{enumerate}
	\item[(1)] 掌握时序逻辑电路的基本分析和设计方法.
	\item[(2)] 理解同步时序逻辑电路和异步时序逻辑电路的区别.
	\item[(3)] 掌握计数器电路设计原理,用硬件描述语言实现指定功能的计数器设计.
	\item[(4)] 学会利用软件仿真实现对数字电路的逻辑功能进行验证和分析.
\end{enumerate}
\section{实验任务}
\begin{enumerate}
	\item[(1)] 使用两个未经译码处理的数码管显示计数,手动单次时钟进行计数,到59的时候回到00。
	\item[(2)] 使用实验平台上的1MHz时钟,将计数器改成秒表,并使用开关控制秒表的启动、暂停。
\end{enumerate}
	
\section{集成“手动”与“秒表”版本}
支持“手动clk/秒表clk/暂停/复位”功能。
\paragraph*{}
我将两个部分功能实现了集成,$mode = '0'$的时候是手动模式,可以手动按下clk实现计数器的加一。$mode='1'$是秒表模式,可以开启秒表模式,在当前计数基础上进行每1s一次的计数。在实现过程中,我使用了大量“元件例化”的方法,并且没有使用"+"。本部分代码位于./counter下。

\subsection{D触发器的实现}
践行模块化设计的思路,我实现了D触发器,实现如下:
\begin{lstlisting}[language={VHDL}, title={d\_ff.vhd}]
-- 异步复位D触发器 --
entity d_ff is
	port(
		clk: in std_logic;
		rst: in std_logic;
		pause: in std_logic;
		d: in std_logic;
		q: out std_logic;
		nq: out std_logic
	);
end d_ff;

architecture bhv of d_ff is
begin
	process(clk, rst) begin
		if rst = '1' then
			q <= '0';
			nq <= '1';
		elsif clk'event and clk='1' then
			if pause='0' then
				q <= d;
				nq <= not d;
			else --维持不变
				q <= not d;
				nq <= d;
			end if;
		end if;
	end process; 
end bhv;
\end{lstlisting}

\subsection{计时器模块}
为了实现“计时”与“计数”的解耦,我将计时器单独作为一个模块。该模块通过记录时钟上升沿个数,每个1M次时钟周期产生一次脉冲。在1MHz的时钟频率下,该模块将产生每1s一次的脉冲,代码如下:
\begin{lstlisting}[language={VHDL}, title={clkpersecond.vhd}]
-- 产生秒表序列(时钟频率1MHz: 产生1s的周期序列) --
entity clkpersecond is
	port(
		clk: in std_logic;
		rst: in std_logic;
		clkps: out std_logic
	);
end clkpersecond;

architecture bhv of clkpersecond is
	signal outclk: std_logic := '0';
	signal count: integer := 0;
	constant FREQUENCY: integer := 500000;--设置时钟频率1MHz(即1000000/2)
begin
	clkps <= outclk;
	process(clk, rst) begin
		if rst = '1' then
			count <= 0;
			outclk <= '0';
		elsif clk'event and clk='1' then
			if count < FREQUENCY then
				count <= count + 1;
			else
				count <= 0;
				outclk <= (not outclk);--跳变
			end if;
		end if;
	end process;
end bhv;
\end{lstlisting}
\subsection{4位加法计数器}
在译码处理前,由于第一位表示0-9的数字,第二位表示0-5的数据,所以需要4位二进制来表示。为了进一步解耦,我实现了用四位二进制数表示计数个数的四位计数器。该模块根据d触发器实现时序逻辑的设计,“十分底层”地实现了计数器,而不是使用简单地+1。具体实现如下:
\begin{lstlisting}[language={VHDL}, title={fourbitcounter.vhd}]
-- 四位加法计数器 --
entity fourbitcounter is
	port(
		clk: in std_logic;
		rst: in std_logic;
		pause: in std_logic;
		qvec: out std_logic_vector(3 downto 0)
	);
end fourbitcounter;

architecture bhv of fourbitcounter is
	component d_ff
		port(
			clk: in std_logic;
			rst: in std_logic;
			pause: in std_logic;
			d: in std_logic;
			q: out std_logic;
			nq: out std_logic
		);
end component;
	signal q_buf: std_logic_vector(3 downto 0);
	signal nq_buf: std_logic_vector(3 downto 0);
begin
	--使用4个d触发器实现4位二进制的自增
	u0: d_ff port map(clk=>clk, rst=>rst, pause=>pause, d=>nq_buf(0), q=>q_buf(0), nq=>nq_buf(0) );
	u1: d_ff port map(clk=>nq_buf(0), rst=>rst, pause=>pause, d=>nq_buf(1), q=>q_buf(1), nq=>nq_buf(1) );
	u2: d_ff port map(clk=>nq_buf(1), rst=>rst, pause=>pause, d=>nq_buf(2), q=>q_buf(2), nq=>nq_buf(2) );
	u3: d_ff port map(clk=>nq_buf(2), rst=>rst, pause=>pause, d=>nq_buf(3), q=>q_buf(3), nq=>nq_buf(3) );
	qvec <= q_buf;
end bhv;
\end{lstlisting}
\subsection{计数器}
有了上面实现的四位计数器和计时器模块,可以很方便地实现计数器了。该模块是顶层实体,通过计时器识别时钟信号并给出1s一次的脉冲,之后四位计数器实现一次自增,最终将输出结果经过译码器(decoder模块与“点亮数字人生”中的模块相同,在此不予赘述)进行显示。代码如下:
\begin{lstlisting}[language={VHDL}, title={counter.vhd}]
-- 计数器(用数码管显示,支持手动clk/秒表clk/暂停/复位 功能, 60进制) --
entity counter is
	port(
		clk: in std_logic;
		rst: in std_logic;
		pause: in std_logic;
		mode: in std_logic;--mode = '1'的时候是秒表,否则是手动加
		highcode: out std_logic_vector(6 downto 0);
		lowcode: out std_logic_vector(6 downto 0)
	);
end counter;

architecture bhv of counter is
	component decoder--译码器
		port(
			bit_4_vec: in std_logic_vector(3 downto 0);
			bit_7_vec: out std_logic_vector(6 downto 0)
		);
	end component;
	component fourbitcounter--四位加法计数器
		port(
			clk: in std_logic;
			rst: in std_logic;
			pause: in std_logic;
			qvec: out std_logic_vector(3 downto 0)
		);
	end component;
	component clkpersecond--产生1s一个上升沿的周期信号
		port(
			clk: in std_logic;
			rst: in std_logic;
			clkps: out std_logic
		);
	end component;
    signal clockUsing: std_logic;--当前使用的clk, 用于区分手动和秒表
	signal clockPerSecond: std_logic;--每1s一次上升沿的clk
	signal tempresetLow: std_logic := '0';
	signal tempresetHigh: std_logic := '0';
	signal resetLow: std_logic := '0';--低位复位
	signal resetHigh: std_logic := '0';--高位复位
	signal outputLow: std_logic_vector(3 downto 0);--低位输出(4位二进制)
	signal outputHigh: std_logic_vector(3 downto 0);--高位输出(4位二进制)
begin
	clockUsing <= clockPerSecond when mode = '1' else clk;
	getclk:clkpersecond port map(clk=>clk, rst=>rst, clkps=>clockPerSecond);
	u1:fourbitcounter port map(clk=>clockUsing, rst=>resetLow, pause=>pause, qvec=>outputLow );
	u2:fourbitcounter port map(clk=>resetLow, rst=>resetHigh, pause=>pause, qvec=>outputHigh );
	decoder1: decoder port map(bit_4_vec=>outputLow, bit_7_vec=>lowcode);
	decoder2: decoder port map(bit_4_vec=>outputHigh, bit_7_vec=>highcode);
	tempresetLow <= '1' when outputLow = "1010" else '0';--到10进1   
	tempresetHigh <= '1' when outputHigh >= "0110" else '0';--到60归零
	resetLow <= rst or tempresetLow; resetHigh <= rst or tempresetHigh;
end bhv;
\end{lstlisting}

\section{仿真结果}
使用Quartus的ModelSim进行仿真(附"\textbf{./counter/Waveform.vwf}"文件).
进行功能仿真的结果如下,第一幅图展示了包含“暂停”和“复位”功能。输出时经过译码后的7位二进制数。第二幅图展示了到达59之后自动回到0的过程。
\begin{figure}[htb]
	\centering
	\includegraphics[width=1.0\textwidth]{simulation1.png}
	\caption{包含“暂停”和“复位”功能}
\end{figure}
\begin{figure}[htb]
	\centering
	\includegraphics[width=1.0\textwidth]{simulation2.png}
	\caption{到达59之后自动回到0}
\end{figure}
\clearpage
\paragraph*{}
下图是时序仿真的结果。展示了包含“暂停”和“复位”功能。
\begin{figure}[htb]
	\centering
	\includegraphics[width=1.0\textwidth]{simulation4.png}
	\caption{时序仿真}
\end{figure}
\section{JieLab运行结果(附录屏)}
我将代码放在了JieLabs实验平台上进行了验证,下面是运行时的截图,在目录下有"\textbf{counter-manual.mp4}"(手动计数)和"\textbf{conuter-auto.mp4}"(秒表功能),视频中还展示了“暂停”,“复位”等功能。您可以更直观地检查我的实现效果。同时附件中提供了从JieLabs存档的“counter-manual.json”和“conuter-auto.json”。
\paragraph*{}
注:左下角板块的第一个开关代表“mode”,mode = 1时是秒表模式,mode = 0是手动模式。第二个开关代表“暂停”。
\begin{figure}[htb]
	\centering
	\subfigure{
		\begin{minipage}[t]{0.48\linewidth}
			\centering
			\includegraphics[width=\textwidth]{counter1.png}
			\caption{手动模式-截图}
		\end{minipage}
	}
	\subfigure{
		\begin{minipage}[t]{0.45\linewidth}
			\centering
			\includegraphics[width=\textwidth]{counter2.png}
			\caption{秒表模式-截图}
		\end{minipage}
	}
\end{figure}
\section{实验总结}
\paragraph*{}
本次实验我采用了大量的“元件例化”方法,在极大程度上实现了“解耦”与模块化设计,我的VHDL编程能力得到了极大提升。我自己实现了d触发器,进而使用时序逻辑的方式实现了计数器,而不是简单地使用"+1"语句,切身体会到了时序逻辑的设计方法和电路特点,没有依赖软件的编译与自动转换,对时序逻辑的增进了理解。同时,我也被硬件所能实现的众多有趣功能所吸引,希望在将来能设计出更多有意义的电路。感谢老师助教在微信群中的耐心指导!
\end{document}