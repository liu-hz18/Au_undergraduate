
\documentclass[UTF8, onecolumn, a4paper]{article}
\usepackage{ctex}
\setlength{\parindent}{2em}
\usepackage{appendix}
\usepackage{geometry}
\usepackage{amsmath, amsthm}
\usepackage{multirow, multicol}
\usepackage{subfigure}
\usepackage{float}
\usepackage{graphicx}
\usepackage{lettrine}
\usepackage{authblk}
\usepackage{indentfirst}
\usepackage{xcolor, fontspec}%用于设置颜色
\usepackage[ruled,vlined]{algorithm2e}
\usepackage{listings}%用于显示代码
\usepackage[colorlinks,
linkcolor=red,
anchorcolor=blue,
citecolor=green
]{hyperref}
\usepackage{tikz}
\usetikzlibrary{trees}
\geometry{left=2.5cm,right=2.5cm,top=2.0cm,bottom=2.0cm}


\title{\textbf{四位加法器: 实验报告}}%———总标题
\author{刘泓尊\quad 2018011446\quad 计84}
%\affil{Department of Computer Science, Tsinghua University}

\begin{document}
\maketitle
\tableofcontents
\lstset{%代码块全局设置
	backgroundcolor=\color{red!3!green!3!blue!3},%代码块背景色为浅灰色
	rulesepcolor= \color{gray}, %代码块边框颜色
	breaklines=true,  %代码过长则换行
	numbers=left, %行号在左侧显示
	numberstyle= \small,%行号字体
	%keywordstyle= \color{red},%关键字颜色
	commentstyle=\color{gray}, %注释颜色
	frame=shadowbox,%用方框框住代码块
	xleftmargin=1em,
	xrightmargin=0em,
	tabsize=5,
	%rulesepcolor=\color{red!20!green!20!blue!20},  %阴影颜色
	keywordstyle={\color{blue!90!}\fontspec{Consolas Bold}},   %关键字颜色
	commentstyle={\color{blue!70!black}\fontspec{Consolas Italic}},   %注释颜色
	stringstyle=\color{orange!100!black}, %字符串颜色
	numberstyle=\color{purple}, %行号颜色
	%basicstyle=\ttfamily, %代码风格
	basicstyle=\fontspec{Consolas},
	showstringspaces=false,          % underline spaces within strings only  
	showtabs=false,
	captionpos=t, %文件标题位置
	flexiblecolumns
}
%\vspace*{20}
\section{File Structure}
\tikzstyle{every node}=[draw=black,thick,anchor=west]
\tikzstyle{selected}=[draw=red,fill=red!30]
\tikzstyle{optional}=[dashed,fill=gray!50]
\begin{center}
	\begin{tikzpicture}
	[
	grow via three points={one child at (0.5,-0.7) and
		two children at (0.5,-0.7) and (0.5,-1.4)},
	edge from parent path={(\tikzparentnode.south)  |-(\tikzchildnode.west)}]
	\node {2018011446}
	child { node {fouradder1}
		child {node {fouradder.vhd}}
		child {node {fulladder.vhd}}
		child {node {halfadder.vhd}}
		child {node {Waveform.vwf}}
		child {node {}}
	}
	child [missing] {}				
	child [missing] {}
	child [missing] {}
	child [missing] {}
	child [missing] {}
	child { node {fouradder2}
		child {node {fouradder.vhd}}
		child {node {fulladder.vhd}}
		child {node {halfadder.vhd}}
		child {node {Waveform.vwf}}
		child {node {}}
	}
	child [missing] {}	
	child [missing] {}				
	child [missing] {}
	child [missing] {}
	child [missing] {}
	child { node {}	}
	child { node {2018011446\_刘泓尊\_四位加法器.pdf}};
	\end{tikzpicture}
\end{center}

\section{实验目的}
\begin{enumerate}
	\item[(1)] 掌握组合逻辑电路的基本分析和设计方法.
	\item[(2)] 理解半加器和全加器工作原理并掌握利用全加器构成不同字长加法器的各种方法.
	\item[(3)] 学习元件例化的方式进行硬件电路设计.
	\item[(4)] 学会利用软件仿真实现对数字电路的逻辑功能进行验证和分析.
\end{enumerate}
\section{实验任务}
\begin{enumerate}
	\item[(1)] 设计实现逐次进位加法器,使用"半加器$\rightarrow$全加器$\rightarrow$逐次进位加法器"方式,进行“功能仿真”并在实验平台上测试。
	\item[(2)] 设计实现超前进位加法器,使用"半加器$\rightarrow$全加器$\rightarrow$超前进位加法器"方式,进行“功能仿真”并在实验平台上测试。
\end{enumerate}
	

\section{逐次进位加法器}
本部分代码位于./fouradder1下
\subsection{1位半加器}
半加器的实现十分简单,只需要处理$S = A \oplus B$, $C = AB$即可,具体实现如下:
\begin{lstlisting}[language={VHDL}, title={halfadder.vhd}]
--一位半加器--
entity halfadder is
	port(
		a, b: in std_logic;
		so, co: out std_logic
	);
end halfadder;

architecture bhv_half of halfadder is
begin
	co <= a and b;
	so <= a xor b;
end bhv_half;
\end{lstlisting}
\subsection{1位全加器}
采用元件例化的方式,使用2个半加器实现全加器。
\begin{lstlisting}[language={VHDL}, title={fulladder.vhd}]
--一位全加器--
entity fulladder is
	port(
		ain, bin, ci: in std_logic;
		s, co: out std_logic
	);
end fulladder;

architecture bhv_full of fulladder is
	component halfadder
		port(
			a, b: in std_logic;
			so, co: out std_logic
		);
	end component;
	signal s1, c1, c2: std_logic;
begin
	u1: halfadder port map(a=>ain, b=>bin, so=>s1, co=>c1);--元件例化
	u2: halfadder port map(a=>s1, b=>ci, so=>s, co=>c2);
	co <= c1 or c2;
end bhv_full;
\end{lstlisting}

\subsection{4位逐次进位加法器}
基于上面实现的1位全加器,可以用串行的策略实现4位逐次进位加法器。
\begin{lstlisting}[language={VHDL}, title={fouradder.vhd}]
--逐次进位加法器,时延9ns左右--
entity fouradder is
	port(
		a, b: in std_logic_vector(3 downto 0);
		cin: in std_logic;
		s: out std_logic_vector(3 downto 0);
		cout: out std_logic
	);
end fouradder;

architecture bhv of fouradder is
	component fulladder
		port(
			ain, bin, ci: in std_logic;
			s, co: out std_logic
		);
	end component;
	signal c: std_logic_vector(3 downto 0);
begin
	u1: fulladder port map(ain=>a(0), bin=>b(0), ci=>cin , s=>s(0), co=>c(0) );--元件例化
	u2: fulladder port map(ain=>a(1), bin=>b(1), ci=>c(0), s=>s(1), co=>c(1) );
	u3: fulladder port map(ain=>a(2), bin=>b(2), ci=>c(1), s=>s(2), co=>c(2) );
	u4: fulladder port map(ain=>a(3), bin=>b(3), ci=>c(2), s=>s(3), co=>cout );
end bhv;
\end{lstlisting}

\section{超前进位加法器}
本部分代码位于./fouradder2下
\subsection{1位半加器}
1位半加器的实现和上面"逐次进位加法器"的1位半加器相同,在此不再赘述。
\subsection{1位全加器}
1位全加器相对于“逐次进位加法器”的1位全加器,主要是增加了“进位产生函数$G_n$”和“进位传递函数$P_n$”.使用$buffer$实现。全加器的实现同样使用了半加器的“例化”。
\begin{lstlisting}[language={VHDL}, title={fulladder.vhd}]
--一位全加器--
entity fulladder is
	port(
		a, b, ci: in std_logic;
		s: out std_logic;
		p, g: buffer std_logic
	);
end fulladder;

architecture bhv_full of fulladder is
	component halfadder
		port(
			ain, bin: in std_logic;
			so, co: out std_logic
		);
	end component;
begin
	u1: halfadder port map(ain=>a, bin=>b, so=>p, co=>g);--半加器例化
	s <= ci xor p;
end bhv_full;
\end{lstlisting}
\subsection{4位超前进位加法器}
4位超前进位加法器的关键是“超前进位单元”,超前进位的输出表达式如下:
\begin{equation*}
\begin{aligned}
	C_0 &= G_0 + P_0C_{-1}\\
	C_1 &= G_1 + P_1G_0 + P_1P_0C_{-1}\\
	C_2 &= G_2 + P_2G_1 + P_2P_1G_0 + P_2P_1P_0C_{-1}\\
	C_3 &= G_3 + P_3G_2 + P_3P_2G_1 + P_3P_2P_1G_0 + P_3P_2P_1P_0C_{-1}
\end{aligned}
\end{equation*}
基于上述逻辑,可以使用元件例化的方式得到$P_i, G_i$,之后处理超前进位部分。代码如下:
\begin{lstlisting}[language={VHDL}, title={fouradder.vhd}]
--Look Ahead Carry four-bit Adder, Time delay: 9ns--
entity fouradder is
	port(
		a, b: in std_logic_vector(3 downto 0);
		cin: in std_logic;
		s: out std_logic_vector(3 downto 0);
		cout: out std_logic
	);
end fouradder;

architecture bhv of fouradder is
	component fulladder
		port(
			a, b, ci: in std_logic;
			p, g, s: out std_logic
		);
	end component;
signal p, g, c: std_logic_vector(3 downto 0);

begin
	u0: fulladder port map(a=>a(0), b=>b(0), ci=>cin, p=>p(0), g=>g(0), s=>s(0));
	u1: for i in 1 to 3 generate--使用生成语句,元件例化
		ux: fulladder port map(a=>a(i), b=>b(i), ci=>c(i-1), p=>p(i), g=>g(i), s=>s(i));
	end generate;
	--超前进位部分--
	c(0) <= g(0) or (p(0) and cin);
	c(1) <= g(1) or (p(1) and g(0)) or (p(1) and p(0) and cin);
	c(2) <= g(2) or (p(2) and g(1)) or (p(2) and p(1) and g(0)) 
				 or (p(2) and p(1) and p(0) and cin);
	c(3) <= g(3) or (p(3) and g(2)) or (p(3) and p(2) and g(1)) 
				 or (p(3) and p(2) and p(1) and g(0)) 
				 or (p(3) and p(2) and p(1) and p(0) and cin);
	cout <= c(3);
end bhv;
\end{lstlisting}

\section{仿真结果}
使用Quartus的ModelSim进行仿真(根目录附"\textbf{Waveform.vwf}"文件),结果如下
\begin{figure}[htb]
	\centering
	%\includegraphics[width=1.0\textwidth]{}
\end{figure}
\section{JieLab运行结果(附录屏)}
我将代码放在了JieLab实验平台上进行了验证,下面是运行时的截图,在根目录下有"\textbf{JieLabs录屏.mp4}",您可以更直观地检查我的实现效果
\begin{figure}[htb]
	\centering
	%\includegraphics[width=0.8\textwidth]{}
\end{figure}
\section{实验总结}
\paragraph*{}
这是我第二次进行CPLD的实验,有了上一次的练习,这次的实验十分顺利,VHDL代码也是一次成形,没有遇到bug。我学到了“元件例化”的方法,极大地提高了硬件设计的简洁性,体会到了模块化设计在电路设计中的重要性。此次实验让我对全加器、超前进位机制等有了实践性的任务,切身体会到了硬件的延时,"理论$\neq$实际"。感谢老师助教在微信群里的耐心指导!

\end{document}